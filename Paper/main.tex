\documentclass[sigconf,authorversion,nonacm]{acmart}
\usepackage{blindtext}
\usepackage{multicol}
\usepackage{parskip}
\usepackage{float}
%\setlength{\columnsep}{1cm}
\title{Rust Safety}
\author{Lucas Hasting}
\affiliation{%
  \institution{University of North Alabama}
  \city{Florence, Al}
  \country{United States}}

\usepackage{listings}
\usepackage{xcolor}

\AtBeginDocument{%
  \providecommand\BibTeX{{%
    Bib\TeX}}}

\settopmatter{printfolios=true}

\definecolor{backcolour}{rgb}{0.95,0.95,0.92}

\lstdefinestyle{mystyle}{
    backgroundcolor=\color{backcolour},   
    basicstyle=\ttfamily\footnotesize,
    breakatwhitespace=false,         
    breaklines=true,                 
    captionpos=b,                    
    keepspaces=true,                 
    numbersep=5pt,                  
    showspaces=false,                
    showstringspaces=false,
    showtabs=false,                  
    tabsize=2
}

\lstset{style=mystyle}
\setcopyright{none}

\begin{document}
\maketitle

\section*{Introduction}
%Broad introduction to the purpose of the paper, including your research question, statement, or hypothesis.
This paper will explore the idea around the safety concerns that exist when programming memory operations in Rust. Rust is a relatively new systems programming language focused on speed and safety [2]. First, the history of Rust is explored and an overview of Rust is given. Then, a review of the literature chosen for the paper is summarized and some security terms are defined for the methodology. The methodology of researching the purpose of the paper is shown. Finally, the paper discusses some questions that could be researched in the future.

\section*{Background}
\subsection*{Historical Background}
%Historical background: A historical overview and impact of the language
Rust was created by Graydon Hoare, he started on the project in 2006 as a passion project and Mozilla started sponsoring the language in 2010. The first compiler for Rust was created in 2012. Rust is known as being a safe, yet efficient, systems programming language, ranked 18 on the TIOBE index in 2020 [2]. 

\subsection*{Rust Overview}
%Language Overview: An overview of the details of the language, including programming capabilities (OOP, imperative, functional), syntax, semantics, and design implementation, such as naming, binding, scope, typing, expressions, control structures, ADTs (built-in and/or buildable), concurrency, etc. This is not an exhaustive list. This section, however, should thoroughly describe the language and will likely be your largest section.

Rust is a systems programming language, that supports many features which will be discussed throughout this section. To start, Rust uses blocks ("\{\}") to represent the scope and a semi-colon (";") to represent the end of an instruction [1], Figure 1 demonstrates this. Also, the keyword fn is used to define a function, and all programs need a main function which is called when the program is executed. Figures 1-17 show demo programs to demonstrate the syntax of the Rust language, these programs and all the tables can be found in the Programs and Tables section. The semantics of what these programs demonstrate will be discussed throughout this section [1]. 

Some Tables of interest in this section are Tables 1, 2, and 5. Table 1 contains the data types in Rust, Table 5 contains the two main types of strings, and Table 5 contains the operators in Rust. Both Tables contain a description of what they represent, and the operators and data types will be discussed throughout the rest of the section. One thing to keep in mind is that the println!() macro is used for output, macro creation is beyond the scope of this paper. To recognize the difference between a macro and a function, recognize that all macros have an "!" before "()" [1].

Another note about println! is that variable cannot be the first parameter, it has to be a string literal. Variables can be substituted in place of "\{\}" where every "\{\}" represents the next variable parameter, this shows up in many example programs, the first being Figure 3 [1].  

Variables in Rust are bound at compile time using the keyword let, it is strongly typed and uses interpretation if a type is not specified. Variables are read-only by default, and adding the mut keyword allows the value to be changed. See Figure 2 for the syntax of declaring a variable in Rust. It shows two ways to bind a type without the use of type interpretation. Next, if, else if, and else statements can be used for control flow. if and else if check a condition (boolean statement) which can be either true or false, if true it will execute the block else if a different condition is true, it will execute the corresponding block, if all statements are false the else block will execute. Next, a match uses the syntax of "case => functionality" and will execute the functionality based on its case, "\_" represents the default case, see Figure 2 for an example of match and conditionals. Also, Loops in Rust can be based on a condition, counter, or nothing at all [1]. 

Moreover, there are a few types of loops in Rust, these include a for each loop which loops through a structure iterating through each item in the structure, a for loop which is counter-controlled, a while loop which loops if a condition (boolean statement) is true, and a loop that loops forever. A for each and for loop can be found in Figure 6, and a while loop and loop can be found in Figure 3. The break and continue keywords help manage the flow of a loop, continue tells the loop to go to the next iteration, ignoring the rest of the loop, and break causes the loop to end [1].

Continuing, the as operator is used to cast a similar type as another, as long as it is allowed. An example where it would not be allowed is converting a negative signed integer to an unsigned integer, an example can be found in Figure 4 [1].

Rust has a few different ways to perform memory operations, it can involve a reference to a variable using a smart pointer, a reference using a raw pointer, or it can be allocated directly to the heap (using either pointer method). A raw pointer is not recommended and is unsafe, the only way to use a raw pointer is by enclosing it in an unsafe block as shown as an example in Figure 5 [1].

Arrays are a collection of items of a specific data type, which the number of items cannot change, while vectors are a collection of items of a specific data type where the number of items can change. They can be displayed using a for each loop, or a counter-controlled for loop, see Figure 6 for an example [1].

In Rust, a slice of a string is a portion of that string, it can include the whole string as that slice. The \&str data type is a reference to a slice, while the String data type is a reference to a string. Enclosing a string using quotes creates a slice, it can be converted into a string using String::from(""). Slices are also a part of arrays and vectors and Figure 7 has a demonstration of this. The index of either structure can create a slice, for example, 1..3 will create a slice that includes indices 1 and 2 (stopping at 3), and 4.. will create a slice from index 4 to the end of the structure [1].

A tuple works the same way in mathematics, it works as a collection of items where a n-tuple has n items. Each index can be retrieved using the "." operator, an example is shown in Figure 8 [1].

In Rust generics are done similarly to other languages by using <VAR> where VAR represents a type placeholder, some features of a type may need to implement features from a trait, this is shown in Figure 9 where the function get\_value takes in any type, displays it and returns it. The generic type needed to be able to implement the std::fmt::Display to use the println! macro [1], keep in mind the example only works if the type used has an implementation of std::fmt::Display.

In Rust, structures like arrays, vectors, and strings are moved to another variable and then dropped from the previous variable it is assigned. Figure 10 shows the problem. In the example, String is chosen and the clone() method is used to create an exact copy instead of assigning the variable [1].

A closure works similarly to a procedure and can be called like a function. variables from a higher level scope are borrowed by default and can be moved to that closure by using the keyword move [5], Figure 11 shows an example.

In Rust, structs are how objects are created, unlike languages like C and C++, a struct has permissions, they are private by default and the keyword pub makes them public. Traits are Rust's version of abstract classes/methods, an example is shown in Figure 13. All methods of a class are implemented in an impl block as shown in Figure 12 [1]. 

When references are used, the rustc compiler may believe the lifetime of the variables is ambiguous and a lifetime parameter is required, and it can happen with functions or classes. The lifetime parameter is done similarly to generics, an example is shown in Figures 12 and 13. The lifetime parameters in those figures are 'a and '\_ [5], which were recommended by the compiler.

Table 3 includes utility traits that are a part of the Rust language.

Enums are state-like variables and contain constants that represent the state. A match statement that takes in an enum and performs operations based on the enum's state is referred to as a pattern, an example enum and pattern can be seen in Figure 14 [1].

Operators in Rust can be overloaded by implementing a method in the trait corresponding to the operator being overloaded, Table 6 shows each trait with the corresponding operator. An example of an operator being overloaded is shown in Figure 15 [1].

Rust uses readers and writers for input and output, this goes outside the scope of the paper, but getting input from STDIN can be useful for writing programs. The std::io library has a read\_line method which takes in a reference to a string and will take input from the keyboard and store the string passed in. The trim method for strings is used to trim off any extra white space. Figure 16 shows an example of taking in input from the user and displaying that output [5]. 

Concurrency in Rust is done using threads, a new thread can be declared by using the thread::spawn method from the std::thread library. thread::spawn takes in a closure and will execute that closure separately from the rest of main, the thread::sleep method is used to make the current thread sleep. Although not needed, the Durration::from\_millis method from the std::time::Duration library can make a machine-readable format for milliseconds which is used along with the concurrency example in Figure 17 [5]. 

To conclude this section, Rust has collections included in the standard library, which are implementations of specific data structures, a list of those can be found in Table 4 [1].

\subsection*{Literature Review}
% Literature Review: An overview of current literature as it relates to your research question.

[1]: This is a textbook on the Rust language. The textbook is a comprehensive guide on the Rust language including the systems features of the language. 

[2]: The paper provides an overview of the Rust language including some historical facts and specific features of the Rust language.

[3]: The paper provides an overview of the security of Rust, it discusses several software vulnerabilities and describes how Rust prevents those vulnerabilities from making a system vulnerable, however, a bad programmer can still make a system vulnerable as shown in the methodology section.  

[4]: The paper describes Rust programs that make use of the unsafe feature, and reasons why a program may use unsafe by looking at projects that make use of the unsafe feature. 

[5]: This is the official Rust documentation, the documentation includes resources on learning the language such as small projects, a few textbooks, and other materials. It also provides a guide to contributing to the Rust compiler called rustc. 

[6]: The paper discusses what the Rust language is, the bugs found in projects using Rust, bugs that can occur when using thread, and how to build a Rust bug detector to avoid writing unsecure code.

\section*{Methodology}
%This section should describe your method for answering your question.
To understand the safety concerns in Rust, software vulnerabilities will be tested in Rust. Figures 18-27 demonstrate common software vulnerabilities attempted in the Rust language safely and unsafely, a description of these vulnerabilities is provided in the following subsection.

\subsection*{Security Definitions}
Below are some software security vulnerabilities and their definitions, these terms are used in the methodology section to answer the research question.

Buffer Overflow: This vulnerability occurs whenever a memory slot is accessed outside of the allocated boundary, and that memory slot is being written to [3].

Buffer Over-read: This vulnerability occurs whenever a memory slot is accessed outside of the allocated boundary, and that memory slot is being read [3].

Use-after-free/Dangling pointer: This vulnerability occurs whenever a pointer variable that has been deallocated is accessed, the pointer is called a dangling pointer [3].

Double Free: This vulnerability occurs whenever a pointer variable is deallocated twice [3].

Uninitialized Memory Access: This vulnerability occurs whenever memory that has not been initialized is accessed [3].

Race Condition: This vulnerability occurs whenever a value is being written and read in memory at the same time, causing the behavior to be undesired [3].

\subsection*{Unsafe Rust}
Unsafe Rust was mentioned briefly in the background section, involving Figure 5. Unsafe Rust can only be used if what the programmer is doing is considered unsafe by the compiler [1]. Unsafe Rust is used by many people to use libraries written for other programs, this can be done using foreign functions which act like interfaces to other languages [4] [1]. It is also used by programmers who are not comfortable with using some of the features in Rust like smart pointers, even though it is safer than raw pointers which requires Rust to be unsafe [4]. Also, if the Rust code is unsafe in a function and the function is used, the Rust program is referred to as interior unsafe [6].

\subsection*{The Research}

First, A buffer overflow and over-read is shown in Figures 18 and 19. In Figure 18, an array with a size of 6 is used as a buffer. When trying to access an item out of bounds, a bound check is done on the array and Rust returns a compile time error when the program is compiled. In Figure 19, Rust tries to access/write an item from an array based on where it is located in memory using the std::ptr library. More specifically, the wrapping\_add method offsets the memory address by the number passed in as an argument and the write method will write a value to a memory address. The over-read and overflow are both successful with undesired behavior occurring [3].

Then, Use-after-free/Dangling Pointer is attempted In Figures 20 and 21. In Figure 20, the value of the variable val is dropped when the function ends, so when the reference to the variable is returned, the value at the reference no longer exists and Rust provides a compile-time error. In Figure 21 we can use a raw pointer instead of a function using unsafe, using the drop\_in\_place function, the value in the referenced variable is dropped, and accessing the dereferenced value provides undesired behavior [3].

Next, Double Free is attempted in Figures 22 and 23. Rust does not allow a reference to be dropped twice, causing Figure 22 to provide a compile-time error. While in the unsafe block, Rust does not have a check for dropping a referenced value [3], executing the code in Figure 23 causes Rust to crash once the value is dropped again, and finished is never displayed on the screen.

Also, Rust does not allow uninitialized memory access, causing both Uninitialized Memory Access attempts to provide a compile-time error [3]. This is shown in Figures 24 and 25 to provide a compile-time error.

Finally, a race condition is attempted in Figures 26 and 27. The process this follows goes beyond the scope of the paper and the example (figures 26 and 27) was taken from the official Rust documentation. So, at a high level, Figure 26 attempts to retrieve a value and display that value to the screen in two separate threads. Rust does not allow 2 mutable references to the same memory address at once and uses a borrowed system to allow the value stored to be accessed once a previous operation is completed [1] [5]. In Figure 27, the same operation is done, but the access of the value being written is accessed by skipping the check that determines if the value is currently being used by another thread and therefore by definition (in the Security Definitions section), causes a race condition to occur [5].  

\section*{Results}
%This section should state the result to the question or show the validity to your statement or hypothesis.
Overall, Rust implements strict safety checks before allowing a program to be executed in order to maintain safe programs, this prevents any unsafe code utilizing memory operations in Rust (outside of the unsafe block) [4]. However, this safety check can be turned off by using the unsafe keyword and will make the program as safe as the programmer allows, and could cause the program to be exposed to one of the many vulnerabilities discussed and should be seldom used.

Therefore, the results of this study show that it is safe to perform memory operations in Rust by default, and it protects against many memory-related vulnerabilities. The memory operations in Rust can become unsafe if the programmer forces it to by using the unsafe keyword. Therefore, the concerns involved with performing memory operations are determined by the programmer and the choices made. If the programmer chooses to use an unsafe block, the programmer should be concerned with being exposed to one of the vulnerabilities mentioned in the methodology, otherwise, Rust will not allow unsafe code.

\section*{Future Work}
%Provide an overview of other interesting research questions related to this language or as an extension of this paper.
Rust has many more topics that could be researched. These include the speed of processing Rust code compared to other languages like C, the suitability of Rust to software projects, and the compatibility between Rust and C++.

\begin{thebibliography}{9}
\bibitem{article}
J. Blandy, J. Orendorff, L. Tindall, \textit{Programming Rust Fast, Safe Systems Development}, 2nd ed. Sebastopol, CA, USA: O’Reilly Media, 2021.
\bibitem{article}
W. Bugden, A. Alahmar, “Rust: The Programming Language for Safety and Performance,” 2022, doi: 10.48550/arXiv.2206.05503
\bibitem{article}
H. Xu, Z. Chen, M. Sun, Y. Zhou, M. Lyu, “Memory-Safety Challenge Considered Solved? An In-Depth Study with All Rust CVEs,” 2021, doi: 10.1145/3466642
\bibitem{article}
V. Astrauskas, C. Matheja, F. Poli, P. Müller, A. Summers, “How do programmers use unsafe Rust?,” 2020, doi: 10.1145/3428204
\bibitem{article}
Rust project. \textit{Rust Documentation}. Accessed: April, 17, 2024. Available: https://doc.Rust-lang.org/beta/
\bibitem{article}
B. Qin, Y. Chen, Z. Yu, L. song, Y. Zhang, “Understanding memory and thread safety practices and issues in real-world Rust programs,” 2020, doi: 10.1145/3385412.3386036 
\end{thebibliography}

\section*{Programs and Tables}

\begin{figure} [H] 
\caption{Hello Word}
    \begin{lstlisting}
fn main() {
    println!("Hello World!");
}
    \end{lstlisting}
\end{figure}

\begin{figure} [H]
\caption{Variables/Conditionals/Match}
    \begin{lstlisting}
fn main() {
    //initialize with a type
    let val1 : i8 = 6;
    let val3 = 7_i8;

    //type inference
    let val2 = 6;
    
    //conditional
    if (val1 == val3){
        println!("Block 1");
    } else if (val1 == val2){
        println!("Block 2");
    } else {
        println!("Block 3");
    }

    switch(val1);
    switch(val2);
    switch(val3);
}

fn switch(val: i8){
    //match statement
    match val {
        6 => println!("Case 1"),
        7 => println!("Case 2"),
        _ => println!("Default")
    }

    return;
}
    \end{lstlisting}
\end{figure}

\begin{figure} [H] 
\caption{While Loops and Loops}
    \begin{lstlisting}
fn main() {

    //both loops display 1, 3, 5

    let mut count = 0_i8;

    while(count != 6){
        count += 1;
        if(count % 2 == 0){
            continue;
        }

        println!("{}", count);
    }

    count = 0_i8;
    
    loop{
        if(count == 6){
            break;
        }

        count += 1;

        if(count % 2 == 0){
            continue;
        }

        println!("{}", count);
    }

}
    \end{lstlisting}
\end{figure}

\begin{figure} [H]
\caption{as operator}
    \begin{lstlisting}
fn main(){
    let val1 = 5_u8;
    let val2 = val1 as u16;

    println!("Value: {}", val2);
}
    \end{lstlisting}
\end{figure}

\begin{figure} [H] 
\caption{Memory related operations}
    \begin{lstlisting}
fn main(){
    //declare mutable variable
    let mut val = 5_i8;

    //pass in reference to the function
    read(&val);

    //pass in mutable reference to the function
    change(&mut val);
    read(&val);

    //allocate value in heap as a smart pointer
    let _v = Box::new(val);

    //raw pointer, unsafe
    unsafe {
        let _bad_val: *mut i8 = &mut val as *mut i8; 
        println!("{}", *_bad_val);
    }
}

//mutable reference to a variabel
fn change(val: &mut i8){
    *val = *val + 1;
}

//Read-only reference to a variable
fn read(val: &i8){
    println!("val is {}", *val);
}
    \end{lstlisting}
\end{figure}

\begin{figure} [H] 
\caption{Arrays and Vectors}
    \begin{lstlisting}
fn main(){
    //create and edit an array
    let mut arr1: [u16; 6] = [3, 2, 7, 4, 11, 13];
    arr1[0] = 4;

    //set all elements a specific value, t has 10000 0s 
    let _arr2 = [0; 10000];

    //initialize the vector with values
    let vec1 = vec![3, 2, 7, 4, 11, 13];

    //create empty vector and push values onto it
    let mut vec2 = Vec::new();
    vec2.push(3);
    vec2.push(4);
    vec2.push(5);

    println!("ARRAY1 1:");
    //print array
    for i in arr1 {
        println!("{}", i);
    }

    println!("Vector 1:");
    //print vector
    for i in vec1 {
        println!("{}", i);
    }

    println!("Vector 2:");
    //another way to print 
    for i in 0..vec2.len() {
       println!("{}", vec2[i]);
    }
}
    \end{lstlisting}
\end{figure}

\begin{figure} [H] 
\caption{Slicing}
    \begin{lstlisting}
fn main(){
    //initilize the vector with values
    let vec1 = vec![3, 2, 7, 4, 11, 13];

    //create slices, works with arrays and vector
    let slice1: &[u16] = &vec1[1..3];
    let slice2: &[u16] = &vec1[4..];

    //print the slices
    println!("Slice 1:");

    for i in slice1 {
        println!("{}", i);
    }

    println!("Slice 2:");

    for i in slice2 {
        println!("{}", i);
    }
}
    \end{lstlisting}
\end{figure}

\begin{figure} [H] 
\caption{Tuples}
    \begin{lstlisting}
fn main() {
    let tup = (4, 5, 6);
    println!("{}",tup.0);
    println!("{}",tup.1);
    println!("{}",tup.2);
}
    \end{lstlisting}
\end{figure}

\begin{figure} [H] 
\caption{Generics}
    \begin{lstlisting}
fn main(){
    let mut val1 = 5_i8;
    let mut val2: String = "Hello".to_string();

    //call generic functions
    val1 = get_value::<i8>(val1);
    val2 = get_value::<String>(val2);

    println!("{} {}",val1, val2)
}

//define generic function
fn get_value<T: std::fmt::Display>(value: T) -> T{
    println!("{}", value);
    return value;
}
    \end{lstlisting}
\end{figure}

\begin{figure} [H] 
\caption{Move Process}
    \begin{lstlisting}
fn main(){
    //printing s causes an error, the value has been moved to k
    let mut s = "John Doe".to_string();
    let k = s;
    println!("k: {}, s: {}", k, s) 

    //problem fixed
    let s = "John Doe".to_string();
    let k = s.clone();
    println!("k: {}, s: {}", k, s) 
}
    \end{lstlisting}
\end{figure}

\begin{figure} [H] 
\caption{Closures [5]}
    \begin{lstlisting}
use  std::mem;

fn main() {
    let mut count = 0;

    //closure borrows count
    let mut add_one = || {
        count += 1;
        println!("`count`: {}", count);
    };

    add_one();

    //closure moves count
    let mut count2 = Box::new(0);

    let mut add_one_move = move || {
        *count2 += 1;
        println!("`count2`: {}", *count2);
        mem::drop(count2);
    };

    add_one_move();
}
    \end{lstlisting}
\end{figure}

\begin{figure} [H] 
\caption{Structs}
    \begin{lstlisting}
fn main(){
    //struct object definition
    pub struct Animal<'a> {
        pub name: &'a str,
        action: &'a str
    }

    //struct object implementation
    impl Animal<'_> {
        pub fn talk(&mut self) {
            println!("{}", self.action);
        }
    }

    //create the struct object
    let mut dog = Animal { name: "Joe", action: "Bark"};
    println!("{}",dog.name);
    dog.talk();
}
    \end{lstlisting}
\end{figure}

\begin{figure} [H] 
\caption{Traits}
    \begin{lstlisting}
fn main(){
    //trait definition
    trait AnimalMethods {
        fn talk(&mut self);
    }

    //struct object definition
    pub struct Animal<'a> {
        pub name: &'a str,
        action: &'a str
    }

    //implement the trait for the Animal class
    impl AnimalMethods for Animal<'_> {
        fn talk(&mut self) {
            println!("{}", self.action);
        }
    }

    //create the struct object
    let mut dog = Animal { name: "Joe", action: "Bark"};
    println!("{}",dog.name);

    //call the method implemented by the trait
    dog.talk();
}
    \end{lstlisting}
\end{figure}

\begin{figure} [H] 
\caption{Enums/Patterns}
    \begin{lstlisting}
fn main() {
    //create enum
    let mut light : Light = Light::RED;

    //use the pattern
    println!("{}",pattern(light));

    //change the state
    light = Light::YELLOW;
    
    println!("{}",pattern(light));
}

//define enum
enum Light {
    RED,
    YELLOW,
    GREEN
}
 
fn pattern(light: Light) -> String{
    //define pattern
    match light {
        Light::RED => String::from("RED"),
        Light::YELLOW => String::from("YELLOW"),
        Light::GREEN => String::from("GREEN")
    }
}
    \end{lstlisting}
\end{figure}

\begin{figure} [H] 
\caption{Operator Overloading}
    \begin{lstlisting}
//add trait
use std::ops::Add;

fn main() {
    //test Fraction class with the overloaded + operator
    let fraction1 = Fraction { numerator : 5 , denomonator : 8};
    let fraction2 = Fraction { numerator : 5 , denomonator : 8};
    let fraction3 = fraction1 + fraction2;
    println!("{}/{}", fraction3.numerator, fraction3.denomonator)
}

//create basic Fraction class
struct Fraction {
    pub numerator : i16,
    pub denomonator: i16
}

//implement the add trait for the Fraction class
impl Add<Fraction> for Fraction{
    type Output = Fraction;

    fn add(mut self, rhs: Fraction) -> Fraction {
        self.numerator = self.numerator * rhs.denomonator;
        self.numerator = self.numerator + (rhs.numerator * self.denomonator);
        self.denomonator = self.denomonator * rhs.denomonator;

        return self;
    }
}
    \end{lstlisting}
\end{figure}

\begin{figure} [H] 
\caption{Input/Output from command line [5]}
    \begin{lstlisting}
use std::io;

fn main() {
    //set empty string
    let mut input = String::new();

    //read input
    io::stdin().read_line(&mut input);

    //trim input
    println!("You typed: {}", input.trim());
}
    \end{lstlisting}
\end{figure}

\begin{figure} [H] 
\caption{Concurrency [5]}
    \begin{lstlisting}
use std::thread;
use std::time::Duration;

fn main() {
    //create a new thread
    thread::spawn(|| {
        for i in 1..10 {
            println!("hi number {} from the spawned thread!", i);

            //sleep
            thread::sleep(Duration::from_millis(1));
        }
    });

    for i in 1..5 {
        println!("hi number {} from the main thread!", i);
        thread::sleep(Duration::from_millis(1));
    }
}
    \end{lstlisting}
\end{figure}

\begin{figure} [H] 
\caption{Safe Buffer Overflow/Over-read}
    \begin{lstlisting}
fn main(){
    let mut buff: [u16; 6] = [3, 2, 7, 4, 11, 13];

    //buffer over-read
    println!("{}",buff[6]);

    //buffer over-flow
    buff[6] = 5;
}
    \end{lstlisting}
\end{figure}

\begin{figure} [H] 
\caption{Unsafe Buffer Overflow/Over-read}
    \begin{lstlisting}
use std::ptr;

fn main(){
    let mut buff = [3, 2, 7, 4, 11, 13];

    unsafe {
        let ptr: *mut i32 = buff.as_mut_ptr();

        //buffer over-read
        println!("{}",*(ptr.wrapping_add(6)));

        //buffer overflow
        ptr::write(ptr.wrapping_add(6), 5);
    }
}
    \end{lstlisting}
\end{figure}

\begin{figure} [H] 
\caption{Safe Use-after-free/Dangling Pointer}
    \begin{lstlisting}
fn main(){
    println!("{}", drop());
}

fn drop() -> &'static String{
    let val = String::from("Rust");
    return &val;
}
    \end{lstlisting}
\end{figure}

\begin{figure} [H] 
\caption{Unsafe Use-after-free/Dangling Pointer}
    \begin{lstlisting}
use std::ptr::drop_in_place;

fn main(){
    let val: *mut String = Box::into_raw(Box::new(String::from("Rust")));

    unsafe {
        println!("{}", *val);
        drop_in_place(val);
        println!("{}", *val);
    }
}
    \end{lstlisting}
\end{figure}

\begin{figure} [H] 
\caption{Safe Double Free}
    \begin{lstlisting}
fn main(){
    let val: String = String::from("Rust");
    std::mem::drop(val);
    std::mem::drop(val);
}
    \end{lstlisting}
\end{figure}

\begin{figure} [H] 
\caption{Unsafe Double Free}
    \begin{lstlisting}
use std::ptr::drop_in_place;

fn main(){
    unsafe {
        let val: *mut String = Box::into_raw(Box::new(String::from("Rust")));
        println!("{}", *val);
        drop_in_place(val);
        drop_in_place(val);
        println!("finished");
    }
}
    \end{lstlisting}
\end{figure}

\begin{figure} [H] 
\caption{Safe Uninitialized Memory Access}
    \begin{lstlisting}
fn main(){
    let val: i8;
    println!("{}", val);
}
    \end{lstlisting}
\end{figure}

\begin{figure} [H] 
\caption{Unsafe Uninitialized Memory Access}
    \begin{lstlisting}
fn main(){
    unsafe {
        let val: *mut i32;
        println!("{}", *val);
    }
}
    \end{lstlisting}
\end{figure}

\begin{figure} [H] 
\caption{Safe Race Condition [5]}
    \begin{lstlisting}
use std::thread;
use std::sync::atomic::{AtomicUsize, Ordering};
use std::sync::Arc;

let data = vec![1, 2, 3, 4];
let idx = Arc::new(AtomicUsize::new(0));
let other_idx = idx.clone();

thread::spawn(move || {
    other_idx.fetch_add(10, Ordering::SeqCst);
});

println!("{}", data[idx.load(Ordering::SeqCst)]);
    \end{lstlisting}
\end{figure}

\begin{figure} [H] 
\caption{Unsafe Race Condition [5]}
    \begin{lstlisting}
use std::thread;
use std::sync::atomic::{AtomicUsize, Ordering};
use std::sync::Arc;

let data = vec![1, 2, 3, 4];

let idx = Arc::new(AtomicUsize::new(0));
let other_idx = idx.clone();

thread::spawn(move || {
    other_idx.fetch_add(10, Ordering::SeqCst);
});

if idx.load(Ordering::SeqCst) < data.len() {
    unsafe {
        println!("{}", data.get_unchecked(idx.load(Ordering::SeqCst)));
    }
}
    \end{lstlisting}
\end{figure}

\begin{table} [H]
\caption{Data Types in Rust [1]}
    \begin{tabular}{ |c|c| }
    \hline
    Type & Description \\
    \hline
    i8 & 8 bit signed integer \\
    \hline
    i16 & 16 bit signed integer \\
    \hline
    i32 & 32 bit signed integer \\
    \hline
    i64 & 64 bit signed integer \\
    \hline
    i128 & 128 bit signed integer \\
    \hline
    u8 & 8 bit unsigned integer \\
    \hline
    u16 & 16 bit unsigned integer \\
    \hline
    u32 & 32 bit unsigned integer \\
    \hline
    u64 & 64 bit unsigned integer \\
    \hline
    u128 & 128 bit unsigned integer \\
    \hline
    isize & signed integer (32 or 64 bit depending on the system) \\
    \hline
    usize & unsigned integer (32 or 64 bit depending on the system)   \\
    \hline
    f32 & 32 bit floating point number \\
    \hline
    f64 & 64 bit floating point number \\
    \hline
    bool & 1 bit, true or false \\
    \hline
    char & 32-bit unicode character \\
    \hline
    String & dynamically sized string of characters \\
    \hline
    \&str & reference to a non owning string pointer \\
    \hline
    \end{tabular}
\end{table}

\begin{table} [H]
\caption{String Types in Rust [1]}  
    \begin{tabular}{ |c|c|c| }
    \hline
    Type & Description & Example \\
    \hline
    Byte String & string of bytes, no Unicode characters & b"get" \\
    \hline
    String Literal & text wrapped in quotes & "string" \\
    \hline
    \end{tabular}
\end{table}

\begin{table} [H] 
\caption{Utility Traits in Rust [1]} 
    \begin{tabular}{ |c|c| }
    \hline
    Type & Description \\
    \hline
    Drop & Destructors \\
    \hline
    Sized & Trait for types with a fixed size \\
    \hline
    Clone & Trait for types that support cloning their value \\
    \hline
    Copy & Clone trait with a byte-by-byte copy \\
    \hline
    Deref and DerefMut & Trait for smart pointers \\
    \hline
    Default & Trait for values with a sensible default value \\
    \hline
    AsRef and AsMut & Conversion Traits (reference to mutable) \\
    \hline
    Borrow and BorrowMut & Consistent version of AsRef/AsMut \\
    \hline
    From and Into & Type conversion trait \\
    \hline
    TryFrom and TryInto & Type conversion trait that could fail \\
    \hline
    ToOwned & Conversion Traits (reference to owned) \\
    \hline
    \end{tabular}
\end{table}

\begin{table} [H]
\caption{Collections in Rust [1]}
    \begin{tabular}{ |c|c| }
    \hline
    Vec<T> & Vector \\
    \hline
    VecDeque<> & Deque \\
    \hline
    LinkedList<T> & Linked List \\
    \hline
    BinaryHeap<T> & Binary Heap \\
    \hline
    HashMap<T> & Hash Map \\
    \hline
    BTreeMap<T> & Binary Tree Map \\
    \hline
    HashSet<T> & Hash Set \\
    \hline
    BTreeSet<T> & Binary Tree Set \\
    \hline
    \end{tabular}
\end{table}

\begin{table} [H]
\caption{Operators in Rust [1]}
    \begin{tabular}{ |c|c| }
    \hline
    Type & Description \\
    \hline
    + & addition \\
    \hline
    - & subtraction \\
    \hline
    * & multiplication \\
    \hline
    / & division \\
    \hline
    \% & modulus division \\
    \hline
    = & assignment \\
    \hline
    << & bit shift left \\
    \hline
    >> & bit shift right \\
    \hline
    \& & bitwise and \\
    \hline
    | & bitwise or \\
    \hline
    < & less than \\
    \hline
    <= & less than or equal to \\
    \hline
    > & greater than \\
    \hline
    >= & greater than or equal to \\
    \hline
    == & equal \\
    \hline
    != & not equal \\
    \hline
    \&\& & and \\
    \hline
    || & or \\
    \hline
    \end{tabular}
\end{table}

\begin{table} [H]
\caption{Operator Types in Rust [1]}
    \begin{tabular}{ |c|c| }
    \hline
    Type & Operator \\
    \hline
    std::ops::Neg & - \\
    \hline
    std::ops::Not & ! \\
    \hline
    std::ops::Add & + \\
    \hline
    std::ops::Sub & - \\
    \hline
    std::ops::Mul & * \\
    \hline
    std::ops::Div & / \\
    \hline
    std::ops::Rem & \% \\
    \hline
    std::ops::BitAnd & \& \\
    \hline
    std::ops::BitOr & | \\
    \hline
    std::ops::BitXor & \^{}\\
    \hline
    std::ops::Shl & << \\
    \hline
    std::ops::Shr & >> \\
    \hline
    std::ops::AddAssign & += \\
    \hline
    std::ops::SubAssign & -= \\
    \hline
    std::ops::MulAssign & *= \\
    \hline
    std::ops::DivAssign & /= \\
    \hline
    std::ops::RemAssign & \%= \\
    \hline
    std::ops::BitAndAssign & \&= \\
    \hline
    std::ops::BitOrAssign & |= \\
    \hline
    std::ops::BitXorAssign & \^{}= \\
    \hline
    std::ops::ShlAssign & <<= \\
    \hline
    std::ops::ShrAssign & >>= \\
    \hline
    std::cmp::PartialEq & ==, != \\
    \hline
    std::cmp::PartialOrd & <, <=, >, >= \\
    \hline
    std::ops::Index & x[y], \&x[y] \\
    \hline
    std::ops::IndexMut &  x[y] = z,  \&mut x[y] \\
    \hline
    \end{tabular}
\end{table}

\end{document}
